%!TEX TS-program = xelatex
%!TEX encoding = UTF-8 Unicode
% Awesome CV LaTeX Template
%
% This template has been downloaded from:
% https://github.com/posquit0/Awesome-CV
%
% Author:
% Claud D. Park <posquit0.bj@gmail.com>
% http://www.posquit0.com
%
% Template license:
% CC BY-SA 4.0 (https://creativecommons.org/licenses/by-sa/4.0/)
%


%%%%%%%%%%%%%%%%%%%%%%%%%%%%%%%%%%%%%%
%     Configuración
%%%%%%%%%%%%%%%%%%%%%%%%%%%%%%%%%%%%%%
%%% Themes: Awesome-CV
\documentclass[]{awesome-cv}
\usepackage{textcomp}
%%% Override a directory location for fonts(default: 'fonts/')
\fontdir[fonts/]

%%% Configure a directory location for sections
\newcommand*{\sectiondir}{resume/}

%%% Override color
% Awesome Colors: awesome-emerald, awesome-skyblue, awesome-red, awesome-pink, awesome-orange
%                 awesome-nephritis, awesome-concrete, awesome-darknight
%% Color for highlight
% Define your custom color if you don't like awesome colors
\colorlet{awesome}{awesome-red}
%\definecolor{awesome}{HTML}{CA63A8}
%% Colors for text
%\definecolor{darktext}{HTML}{414141}
%\definecolor{text}{HTML}{414141}
%\definecolor{graytext}{HTML}{414141}
%\definecolor{lighttext}{HTML}{414141}

%%% Override a separator for social informations in header(default: ' | ')
%\headersocialsep[\quad\textbar\quad]
    \begin{document}
    
%%%%%%%%%%%%%%%%%%%%%%%%%%%%%%%%%%%%%%
%     Perfil
%%%%%%%%%%%%%%%%%%%%%%%%%%%%%%%%%%%%%%
\begin{center}
	\headerlastnamestyle{Angie Paola} \headerfirstnamestyle{Manrique Ravelo} \\
	\vspace{2mm}
	{\hspace{0.8cm}\faEnvelope\ angie27manrique@gmail.com}  |  {\faMobile\ (+57) 3107669063}  |  {\faMapMarker\ Bogotá, Colombia} 
	\newline {\faLink\ \href{https://www.linkedin.com/in/angie27manrique/}{https://www.linkedin.com/in/angie27manrique/}}
\end{center}
\vspace{-2mm}

\cvsection{Habilidades}
\begin{cventries}
	\vspace{-2mm}
	\cventry
	{}
	{\def\arraystretch{1.15}{\begin{tabular}{ l l }
		Lenguajes:  & {\skill{ Java, C\#, TypeScript, SQL, Python.}} \\
		Frameworks:  & {\skill{ Spring, Angular.}} \\
		Bases de datos:  & {\skill{ PostgreSQL, MySQL, SQLServer, MongoDB.}} \\
		Cloud:  & {\skill{ AWS, Azure.}} \\
		Tecnologías / Herramientas: \hspace{0.05cm} & {\skill{ Docker, Jenkins, GraphQL, Kafka, SonarQube, Maven, npm, Git.}} \\
		Prácticas:  & {\skill{ Agile, Scrum, SOLID Principles, Test-Driven Development, Code Reviews.}} \\
		\end{tabular}}}
	{}
	{}
	{}
\end{cventries}
\vspace{-12mm}

%%%%%%%%%%%%%%%%%%%%%%%%%%%%%%%%%%%%%%
%     Experiencia
%%%%%%%%%%%%%%%%%%%%%%%%%%%%%%%%%%%%%%
\cvsection{Experiencia}
\begin{cventries}
	\cventry
	{Backend Engineer Semisenior}
	{Crehana}
	{Bogotá, Colombia}
	{Feb. 2021 – Jun. 2024}
	{\begin{cvitems}
		\vspace{0.2mm}
		\item {Desarrollé módulos de software innovadores utilizando Java y Spring Framework, asegurando la integridad de datos y resultando en múltiples funcionalidades recurrentes para más de 1,000,000 de transacciones de clientes.}
		\item {Programé más de 10 funcionalidades interactivas en Angular, mejorando la funcionalidad de la plataforma; la integración resultó en un aumento del 15\% en el engagement de usuarios y la herramienta es ahora utilizada por más de 200,000 usuarios activos.}
		\item {Mejoré significativamente el rendimiento del dashboard de reportes mediante la identificación y eliminación de cuellos de botella, logrando una reducción del 50\% en el tiempo de carga.}
		\item {Diseñé y construí APIs RESTful, mejorando la integración de sistemas y la accesibilidad de datos.}
		\end{cvitems}}

	\cventry
	{Web Designer}
	{Ink digital}
	{Bogotá, Colombia}
	{Nov. 2020 – Ene. 2021}
	{\begin{cvitems}
		\vspace{0.2mm}
		\item {Construí diseños innovadores y de alta calidad que constantemente obtuvieron más del 90\% en evaluaciones de rendimiento y accesibilidad.}
		\item {Desarrollé sitios web para clientes utilizando WordPress CMS, logrando un tiempo promedio de carga de 3 segundos o menos.}
		\item {Utilicé Google Ads para posicionar sitios web, generando un aumento del 15\% en tráfico orgánico.}
		\end{cvitems}}

	\cventry
	{Desarrollador y Co-Investigador}
	{Universidad de Cundinamarca}
	{Facatativá, Colombia}
	{Nov. 2019 – Dic. 2020}
	{\begin{cvitems}
		\vspace{0.2mm}
		\item {Creé e implementé una aplicación web para visualizar los resultados gráficos de análisis descriptivos y predictivos utilizando técnicas de minería de datos.}
		\item {Realicé un análisis descriptivo y predictivo de datos del ICFES, que resultó en la creación de 10 indicadores clave para medir las brechas de género en la educación colombiana.}
		\item {Presenté los resultados de la investigación en 3 conferencias nacionales entre 2019 y 2020.}
		\end{cvitems}}
\end{cventries}

\vspace{-2mm}
\cvsection{Proyectos}
\begin{cventries}
	\vspace{-1mm}
	\cventry
	{}
	{AquaFeeder: Sistema de Detección de Peces y Optimización de Alimentación en Tiempo Real \vspace{-5mm}}
	{MATLAB, Arduino, Visión por Computador \vspace{-5mm}}
	{}
	{\begin{cvsectionnormaltext}
		\item{Desarrollé un sistema inteligente utilizando webcam en tiempo real, MATLAB y Arduino para detectar especies de peces y predecir su longitud y porciones de alimento.}
	\end{cvsectionnormaltext}}
	
	\vspace{-4mm}
	\cventry
	{}
	{Brechas de Género en Carreras STEM en las Pruebas Saber 2016 en Colombia \vspace{-5mm}}
	{Revista Pensamiento Udecino \vspace{-5mm}}
	{Nov 6, 2020 \vspace{-5mm}}
	{\begin{cvsectionnormaltext}
		\item{Publiqué artículo de investigación analizando las disparidades de género en la educación superior colombiana.
		\newline \faLink\ \href{https://revistas.ucundinamarca.edu.co/index.php/Pensamiento_udecino/article/view/79}{https://revistas.ucundinamarca.edu.co/index.php/Pensamiento\_udecino/article/view/79}}
	\end{cvsectionnormaltext}}
	
	\vspace{-6mm}
	
\end{cventries}

%%%%%%%%%%%%%%%%%%%%%%%%%%%%%%%%%%%%%%
%     Educación
%%%%%%%%%%%%%%%%%%%%%%%%%%%%%%%%%%%%%%
\vspace{1mm}
\cvsection{Educación}
\begin{cventries}
	\vspace{-3mm}
	\cventry
	{}
	{Universidad de Cundinamarca \vspace{-5mm}}
	{Facatativá, Colombia \vspace{-5mm}}
	{Feb 2016 – Mar 2021 \vspace{-5mm}}
	{\begin{cvsectionnormaltext} 
		\item{Pregrado en Ingeniería de Sistemas}
	\end{cvsectionnormaltext}}
\end{cventries}

%%%%%%%%%%%%%%%%%%%%%%%%%%%%%%%%%%%%%%
%     Certificaciones
%%%%%%%%%%%%%%%%%%%%%%%%%%%%%%%%%%%%%%
\vspace{-7mm}
\subsection*{Certificaciones}
\begin{cventries}
	\vspace{-1mm}
	\cventry
	{}
	{Oracle Cloud Infrastructure 2023 AI Certified Foundations Associate \vspace{-5mm}}
	{Oracle \vspace{-5mm}}
	{2024 \vspace{-5mm}}
	{}
\end{cventries}

\end{document}